\newacronym{ksa}{KSA}{Kantonsschule Alpenquai}
\newacronym{bla}{bla}{bla bla}
\newacronym{idn}{idn}{identification number}

\newglossaryentry{plugin}{name=Plugin, description={Ein Plugin ist ein optionales Modul, welches die Funktionalität einer Softwareanwendung erweitert}}

\newglossaryentry{qgis}{name=QGIS, description={\href{http://www.qgis.org}{QGIS} ist eine benutzerfreundliche Open-Source GIS-Software. QGIS läuft unter Linux, Unix, Mac OSX, Windows und Android. Die Entwicklung von QGIS wurde vom Kanton Solothurn stark unterstützt und gefördert. Heute umfasst QGIS eine grossen Funktionsumfang und ist mit proprietären Produkten wie ArcGIS von ESRI zumindest ebenbürtig. Eine grosse Stärke von QGIS liegt in der Unterstützung einer Vielzahl von Vektor-, Raster- und Datenbankformaten. Durch die dynamische Entwicklergemeinschaft, welche QGIS ständig ausbaut, ist QGIS jedoch für gelegentliche Anwender komplex geworden und benötigt eine Unterstützung und Begleitung.  Die Anwendung von QGIS im Verwaltungsalltag muss deshalb gezielt erfolgen und unterstützt werden}}
