% !TEX TS-program = XeLaTeX
% !TEX encoding = UTF-8 Unicode
\documentclass[11pt,german,aspectratio=169]{beamer}
\usepackage{KSA} 

%_________________________________________________________________________
% Grundfarbe
\def\myColor{blue!80!white!70!green}                % siehe usepackage color         
\setbeamercolor{structure}{fg=\myColor!65!black}    % für Überschriften und bullets  wird noch mit 35% schwarz abgedunkelt

% Befehle für Schriftarten
\newcommand{\thisColor}[1]{{\color{\myColor}#1}}    % Argument wird in der Grundfarbe geschrieben
\newcommand{\grey}[1]{{\color{black!60}#1}}         % Argument wird grau geschrieben
\newcommand{\tttc}[1]{{\tt\thisColor{#1}}}          % Argument wird in der Grundfarbe und in typetext geschrieben
\newcommand{\ttgr}[1]{{\tt\grey{#1}}}               % Argument wird grau und in typetext geschrieben

% Befehle für URLs
\newcommand{\smallurl}[1]{{\scriptsize{\url{#1}}}}
\newcommand{\tinyurl}[1]{{\tiny{\url{#1}}}}


% Schlüsselwörter für listings (hier für LaTeX eingestellt)
\lstset{language=[LaTeX]TeX,frame=single,basicstyle=\tiny\ttfamily,keywordstyle=\color{blue}\bfseries, 
backgroundcolor=\color{black!3}, stringstyle=\ttfamily, 
morekeywords={subsection, subsubsection, maFRAGE, maAutorIn, maOrt, maKlasse, maTitel, maBetreuung, citep} 
}

%_________________________________________________________________________
% WICHTIG: Pfad für die Bilder der Präsentation
\def\BILDER{../Bilder/} % Achtung: kantonLU.png und ksalpenquai.png müssen in diesem Ordner sein!

\title{ \LaTeX\ für Maturaarbeiten an der \\Kantonsschule Alpenquai\\
	%\includegraphics[width=0.8\textwidth]{\BILDER Titelbild.png}
}

\author{Christoph von Arx}
\date{12. \& 26. \,November 2020	}
%_________________________________________________________________________
\begin{document}
\frame{\titlepage}   % Titelfolie

\begin{frame}
\frametitle{Inhalt}  % Inhalt
\tableofcontents
\end{frame}
%_________________________________________________________________________

\section{Einführung} % Die Gliederung bestimmt das Inhaltsverzeichnis
\frame
{
	\frametitle{Was ist \LaTeX ?}
	\LaTeX\ ist ein Textsatzsystem, das hohe typographische Ansprüche erfüllt.

\begin{center}
		\parbox[c]{0.98\textwidth}
		{\begin{description}
	\item[Eingabe]ist ein reines Text-Dokument in einem Text-Editor mit inhaltlichem Text und \LaTeX-Befehlen.
	\item[Verarbeitung]erfolgt durch den Befehl \enquote{Setzen}.
	\item[Ausgabe]ist ein formatiertes PDF-Dokument.
\end{description}
}
\end{center}
}

\frame
{
	\frametitle{Verarbeitungsschema}
	 	\begin{center}
	 \includegraphics[width=0.95\textwidth]{\BILDER latexSchema_blau}
	 	\end{center}
		\vfill
{\tiny (Quelle: \url{http://latex.tugraz.at/latex/tutorial})}
}


\frame
{
  \frametitle{Warum \LaTeX\ und nicht MS Word?}
\begin{itemize}
	\item professioneller Textsatz
	\item klar und einheitlich strukturierte Dokumente
	\item für mathematische Gleichungen überragend
	\item standardisierte Verwendung von Quellen  \\$ \Rightarrow$ Zusammenarbeit mit Quellen-Datenbanken
	\item sehr einfache Handhabung von Verweisen und Fussnoten
	\item Inhalt und Form sind getrennt \\$ \Rightarrow$ Konzentration auf Inhalt, Vorlage definiert Form
	\item an einigen Hochschulinstituten \emph{der} Standard
	\item open source, lauffähig auf Windows, OS X, Linux
	\item modular und erweiterbar: Bücher, Chemie, Musikpartituren, Schach, \dots
\end{itemize}
}

\frame
{
	\frametitle{Hindernisse}
\begin{itemize}
	\item braucht Umgewöhnung
	\item beachtliche Einstiegshürde
\end{itemize}


}

\section{in vier Schritten zum Ziel}
\frame
{
	\frametitle{Schritt 1: Installation von TeX Live}
	
	Homepage des Pakets Maturaarbeit KSA: \\
	\vspace{2ex}
	\url{https://gitlab.com/c.v.a/maturaarbeit-KSA} \\
	\vspace{1ex}	
	Link \ttgr{Readme} anklicken.\\
	\vspace{2ex}
	Readme enthält
	
	\begin{itemize}
		\item Link zu \ttgr{FAQ Wiki} (im Aufbau begriffen)
		\item Kurzanleitung für die Installation von \tttc{TeX Live}\ mit den benötigten Links
		\item Link zum Herunterladen der eigentlichen Vorlage
		\item Einstieg zum Schreiben
	\end{itemize}
	
}
\frame
{
	\frametitle{Schritt 1a: Einstellung Texteditoren}
	\begin{itemize}
		\item Encoding: unbedingt UTF-8 als Standard setzen.
		\item Rechtschreibung: siehe Anleitung \href{run:Installation-Rechtschreibung.pdf}{\tttc{Installation-Rechtschreibung.pdf}}
		\item Vervollständigung: siehe Anleitung \href{run:Vervollständigung.pdf}{\tttc {Vervollständigung.pdf}}\\
		(der schwarze Punkt \textbullet\ wird durch Alt-Tab $[$OS X$]$ oder Ctrl-Tab $[$sonst$]$ erreicht.)
	\end{itemize}
}
\frame
{
	\frametitle{Schritt 2: Installation der eigentlichen Vorlage}
	
	
	Homepage der Vorlage: \\
	\smallurl{https://gitlab.com/c.v.a/maturaarbeit-KSA/-/archive/master/maturaarbeit-KSA-master.zip}
	
			\begin{description}
 				\item[MA-Vorlage] {\tt{MA.tex}} mit \thisColor{\XeTeX}\ oder \thisColor{\XeLaTeX}\\ zum Laufen bringen
			\end{description}
	\begin{itemize}
		\item Damit ist die Form schon erfüllt und die Note 6 dafür ist garantiert ;-)
		\item Der Fokus ist jetzt zu \SI{100}{\percent} auf dem Inhalt.
	\end{itemize}
}

\frame
{
	\frametitle{Schritt 3: Eigene Maturaarbeit beginnen (Inhaltsverzeichnis)}
	
		\begin{itemize}
			\item Textbausteine setzen
	\lstinputlisting{bsp1.tex}
			
			\item Kapitelüberschriften hinzufügen, weglassen, umbenennen \\
	\lstinputlisting{bsp2.tex}
			\item Unterkapitelüberschriften hinzufügen \\
	\lstinputlisting{bsp3.tex}
			\item Abschnittüberschriften hinzufügen \\
	\lstinputlisting{bsp4.tex}
		\end{itemize}
	
}
\frame
{
	\frametitle{Schritt 4: Eigene Arbeit mit Inhalt füllen}
	\begin{itemize}
		\item Vergleichen mit Beispiel-Arbeit: schauen, wie etwas gemacht wird und nachahmen
		\item bei Problemen in den Anleitungen nachschlagen
		\item die Internetforen bieten oft gute Lösungen zu Problemen mit \LaTeX
		\item bei Schwierigkeiten: ungeniert mail senden an \href{mailto:christoph.vonarx@edulu.ch}{\tt{christoph.vonarx@edulu.ch}}\\
		{\color{green!80!white!70!black} möglichst früh!}
		
	\end{itemize}
	
}

\section{arbeiten mit Quellen}
\frame
{
	\frametitle{Quellen}
	\begin{itemize}
		\item \LaTeX\ arbeitet mit \thisColor{Biber}\ zusammen. Quellen sind standardisiert erfasst:
	\lstinputlisting{bsp5.tex}
	\item Erfassung geschieht mit Datenbankprogrammen:
		\begin{description}
			\item[JabRef] auf allen Plattformen \\
			
		\url{http://jabref.sourceforge.net}
			\item[BibDesk] auf OS X	\\ \url{http://bibdesk.sourceforge.net}		
		\end{description}
	\end{itemize}
	
}
\frame
{
	\frametitle{Beipiel eines Zitats}
	\begin{itemize}
		\item Quelltext \\
			\lstinputlisting{bsp6.tex}
		\item Dokument \\
		 	\begin{center}
		 \includegraphics[width=0.90\textwidth]{\BILDER zitat1}
		 \end{center}
		\item Literaturverzeichnis \\
		 	\begin{center}
		 \includegraphics[width=0.90\textwidth]{\BILDER zitat2}
		 
		 	\end{center}
		
		
	\end{itemize}
}
\frame
{
	\frametitle{Quellen: Hinweis für das Setzen}
	
	Für das Erstellen der Quellen braucht es mehrere Schritte
	\begin{itemize}
		\item Satz mit \thisColor{\LaTeX}: bestimmt die verwendeten Quellen
		\item Satz mit \thisColor{Biber} (holt die Quellen)
		\item Satz mit \thisColor{\LaTeX}: erstellt das Quellenverzeichnis
		\item (eventuell nochmals \thisColor{\LaTeX})
		
	\end{itemize}
}

%\frame
%{
%	\frametitle{auf Wunsch: getrenntes Literatur- \\und Internetquellen-Verzeichnis}
%	\begin{itemize}
%		\item die Unterscheidung erfolgt erst unmittelbar vor Fertigstellung der Arbeit
%		\item eine Kurzanleitung dafür wird auf Wunsch von mir gemacht
%	\end{itemize}
%}

\section{nützliche Links}
\frame
{
	\frametitle{Hilfen}
	\begin{itemize}
		\item 	im Ordner {\texttt Dokumentationen}\\
	\item nützliche Links:
	\begin{itemize}
		\item \tinyurl{https://latex.tugraz.at/latex/tutorial}
		\item \tinyurl{https://www.tobiaspaul.net/dokuwiki/doku.php?id=software:latex:tipps_tricks}
		\item \tinyurl{http://latex.wikia.com/wiki/Main_Page}
	\end{itemize}
	\end{itemize}
	
}
\section{Abschluss}

\frame
{
	\frametitle{Abschluss}
Nach diesen ersten Eindrücken von \LaTeX\ sind Sie hoffentlich motiviert, dieses Textsatzprogramm zu verwenden.

Lassen Sie sich von anfänglichen Schwierigkeiten nicht entmutigen, die Investition zahlt sich aus!

\vspace{2ex}
\thisColor{Ein herzliches Dankeschön geht an}
	\begin{itemize}
		\item Sie für Ihr Interesse.
		\item Herrn Hubert Imhof und Frau Esther Holl (Prorektorin), für die Unterstützung.
		\item Theo von Arx, der mich auf die Idee dieser Vorlage und GitLab gebracht hat.
		\item Maximilian Rothstein und alle Schülerinnen und Schüler, welche wertvolle Rückmeldungen gegeben haben.
	\end{itemize}

\vfill
{\tiny
Nota bene: selbstverständlich wurde auch diese Präsentation mit \LaTeX\ erstellt\dots
}
}

\end{document}
