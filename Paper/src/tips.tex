% !TEX encoding = UTF-8 Unicode
% !TEX root = MA.tex

\section*{Tips}
\subsection*{Cheat Sheet}
Viele gute Tips sind im \href{run:../Dokumentationen/cheat-sheet.pdf}{\ttfamily cheat-sheet.pdf} zu finden.

\subsection*{Anführungs- und Schlusszeichen, Texthervorhebungen}
Mehrfache Leerschläge            im Quelltext haben keine Auswirkungen.

Abschnitte werden durch eine leere Linie erzeugt.

Anführungs- und Schlusszeichen werden im Volksmund auch \enquote{Gänsefüsschen} genannt. 

Ein  \textbf{fett gesetzter Teil} behindert den Lesefluss. Deswegen ist ein \emph{kursiv gesetzter Text} zu bevorzugen.

Eigennamen (zum Beispiel für den Satz von \textsc{Pythagoras}), werden häufig in \textsc{Kapitälchen} gesetzt.
\subsection*{Einfache Textbausteine}

\def\FCL{Fussballclub Luzern}
Der Befehl \begin{verbatim*}\def\FCL{Fussballclub Luzern}\end{verbatim*} 
definiert die Abkürzung \begin{verbatim*}\FCL\end{verbatim*} 
als \enquote{\FCL}.

\subsection*{Tabellen}
Tabellen haben eine grosse Anfangshürde. Die Webseite \url{http://www.tablesgenerator.com} ist sehr nützlich. Tabellen können in Excel erstellt werden und dann auf der Website in \LaTeX -Quellcode umgewandelt werden.

Hier ist eine Tabelle mit Dezimaltabulator:
\begin{table}[htp]
\caption{Messwerte}
\label{tab:messwerte1}
\begin{center}
\begin{tabular}{|d{0}|d{1}|}
\hline
\text{Zeit in (s)} & \text{Temperatur in \SI{}{\celsius}} \\
\hline\hline
0 & 10.5 \\
30 & 15.7 \\
\hline
\end{tabular}
\end{center}
\end{table}

\subsection*{Bilder}
Das Logo unsrer Schule ist in Abbildung~\vref{fig:ksa} zu finden.
\begin{figure}[htb]
	\centering
		\includegraphics{\BILDER ksalpenquai} \\
		\vspace{1ex}
		\includegraphics[width=0.3\textwidth]{\BILDER ksalpenquai}
 	\caption[Logo unserer Schule]{Logo unserer Schule, in Originalgrösse und auf 0.3-fache Textweite gesetzt.\footnotemark}
  \label{fig:ksa}
\end{figure}
	\footnotetext{Beispiel für eine Fussnote in einer Bildbeschreibung.}

\textbf{Wichtiger Hinweis}: der Dateiname eines Bildes darf keine Umlaute enthalten.

\subsection*{Aus Mathematik und Physik}

Wenn ein Mensch mit \SI{1}{\meter\per\second} unterwegs ist, so beträgt seine Geschwindigkeit \SI{3600}{\meter\per\hour}. 
Das ist gleichviel wie \SI{3.6}{\kilo\meter\per\hour} oder \SI{1e9}{\nano\meter\per\second}.

Eine Schokolade kostet \FR{3.20}

Die Formel $F =  m \cdot  g$ ist den Leuten im Maturajahr mit einer Wahrscheinlichkeit von \SI{73.1}{\percent} bekannt. 

Zahlen können so dargestellt werden: \num{0.1234567}. Multipliziert mit \num{1e6} ergibt das \num{123456.7}.

Eine Gleichung ohne Nummer sieht so aus:
\begin{equation*}
c^2 =  a^2 + b^2
\end{equation*}

Eine Gleichung mit Kasten und Nummer sieht so aus:
\begin{equation}
\boxed{
	c^2 =  a^2 + b^2
}
\end{equation}

Eine andere Umgebung für mehrere Gleichungen untereinander ist hier dargestellt:
\begin{align*}
e &= \SI{1.60E-19}{\coulomb} &\text{Elementarladung} \\
m_{e} &=  \SI{9.11E-31}{\kilo\gram} &\text{Ruhemasse des Elektrons}\\
u &=  \SI{1.66e-27}{\kilo\gram} &\text{Atommassen-Einheit} \\
G&= \SI{6.67259E-11}{\meter^3\per(\kilo\gram\second^2)} &\text{Gravitationskonstante} \\
k& = \SI{8.988E9}{\newton\meter^2\per\coulomb^2}= \frac{1}{4 \pi \cdot \epsilon_0}  &\text{Elektrostatik} \\
\epsilon_0&= \SI{8.854E-12}{\coulomb^2\per(\newton\meter^2)} &\text{elektrische Feldkonstante}
\end{align*}

Dasselbe, aber mit nur teilweise nummerierten Gleichungen:
\begin{align}
e &= \SI{1.60E-19}{\coulomb} &\text{Elementarladung} \\
m_{e} &=  \SI{9.11E-31}{\kilo\gram} &\text{Ruhemasse des Elektrons}\notag\\
u &=  \SI{1.66e-27}{\kilo\gram} &\text{Atommassen-Einheit} \\
G&= \SI{6.67259E-11}{\meter^3\per(\kilo\gram\second^2)} &\text{Gravitationskonstante} \\
k& = \SI{8.988E9}{\newton\meter^2\per\coulomb^2}= \frac{1}{4 \pi \cdot \epsilon_0}  &\text{Elektrostatik} \\
\epsilon_0&= \SI{8.854E-12}{\coulomb^2\per(\newton\meter^2)} &\text{elektrische Feldkonstante}
\end{align}

Auch Vektoren können einfach dargestellt werden: 
\begin{equation*}
\vec{a} =  \frac{ \Delta \vec{v} }{ \Delta t} 
\end{equation*}
\subsection*{Geogebra}

Bilder können exportiert werden als \LaTeX -Code.

\begin{pspicture*}(-1.12,-0.72)(8.42,3.48)
\psset{xunit=1.0cm,yunit=1.0cm,algebraic=true,dimen=middle,dotstyle=o,dotsize=3pt 0,linewidth=0.8pt,arrowsize=3pt 2,arrowinset=0.25}
\multips(0,0)(0,1.0){5}{\psline[linestyle=dashed,linecap=1,dash=1.5pt 1.5pt,linewidth=0.4pt,linecolor=lightgray]{c-c}(-1.12,0)(8.42,0)}
\multips(-1,0)(1.0,0){10}{\psline[linestyle=dashed,linecap=1,dash=1.5pt 1.5pt,linewidth=0.4pt,linecolor=lightgray]{c-c}(0,-0.72)(0,3.48)}
\psaxes[labelFontSize=\scriptstyle,xAxis=true,yAxis=true,Dx=1.,Dy=1.,ticksize=-2pt 0,subticks=2]{->}(0,0)(-1.12,-0.72)(8.42,3.48)
\rput[tl](4.66,2.9){$ \SI{1}{\meter\per\second} \hat= \SI{1}{\centi\meter}$}
\end{pspicture*}

\subsection*{Chemie}
Das ist eine Reaktionsgleichung:
\begin{equation}
\ce{CO2 + C -> 2 CO} 
\end{equation}
Im Text können chemische Gleichungen wie \ce{A <--> B} ebenfalls verwendet werden\footnote{Hilfe zu Chemie findet sich im Helpfile mhchem.pdf des Pakets \enquote{mhchem}}.

Etwas komplizierter:
\begin{align*}
  \ce{RNO2 &<=>[+e] RNO2^{-.} \\
       RNO2^{-.} &<=>[+e] RNO2^2-}
\end{align*}

Falls Strukturformeln benötigt werden:\\
\url{https://www.latex-kurs.de/kurse/2017/Kurs2/Teil10/Chemie.pdf}
\begin{equation}
\chemfig{H-C(-[2]H)(-[6]H)-H} \quad\quad \chemfig{*6(=-=-=-)} \quad\quad \chemfig{H-[:52.24]\lewis{1:3:,O}-[::-104.48]H}
\end{equation}

\subsection*{Quellen}

Hier wird eine Webseite~\parencite{maKSA} zitiert. \cite{Theis2014} hat sein Buch geschrieben. Daraus ist speziell ein Fallbeispiel \cite[siehe][Seite 32]{Theis2014} interessant. Näheres ist in \url{https://en.wikibooks.org/wiki/LaTeX/Bibliographies_with_biblatex_and_biber} zu finden. \footnote{Eine Fussnote mit einer Quellenangabe \parencite[siehe][Seite 32]{Theis2014}.}

Hier wird eine URL mit Umlaut zitiert:\cite{Gärtner}.

\subsection*{Glossar, Akronyme}
Hilfe: \url{https://en.wikibooks.org/wiki/LaTeX/Glossary}.

An der \Acrlong{ksa}  ist die Abkürzung  \acrshort{ksa} geläufig.


Der erste Gebrauch von \gls{idn} sieht anders aus als der zweite Gebrauch von \gls{idn}. 

Der Gebrauch von \acrshort{idn} sieht anders aus als der  Gebrauch von \ACRshort{idn}. 

Der erste Gebrauch von \gls{bla} sieht anders aus als der zweite Gebrauch von \gls{bla}. 

Der erste Gebrauch von \gls{qgis} sieht nicht anders aus als der zweite Gebrauch von \gls{qgis}. 

