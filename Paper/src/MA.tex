% !TEX TS-program = XeLaTeX
% !TEX encoding = UTF-8 Unicode
% Version 2019-11-14, C. von Arx

\documentclass[12pt, german]{scrartcl}   % für deutsche MA
%\documentclass[12pt, english]{scrartcl} % für englische MA 

\usepackage{KSA}[2019/11/14]  % KSA Vorlage, enthält die gängigsten Pakete   


%___________________________ Definitionen des Literaturverzeichnisses ________
\usepackage[style=numeric,natbib=true,doi=true,url=true,isbn=true,sorting=none,backend=biber]{biblatex}    % Quellen mit Nummern 
%\usepackage[style=chem-angew,natbib=true,doi=true,url=true,sorting=none,backend=biber,isbn=true]{biblatex} % von der Chemie bevorzugt

%\usepackage[style=ext-authoryear-icomp,natbib=true,doi=true,url=true,isbn=true,backend=biber]{biblatex}    % AutorIn, Jahr 
%\DeclareOuterCiteDelim{parencite}{\bibopenbracket}{\bibclosebracket} % (nur zusammen mit style=ext-authoryear-icomp) aktivieren für eckige Klammern

\addbibresource{Quellen.bib} % Name der Datei, welche die Quellen enthält

%___________________________ Definitionen von Glossar ________________________
%\usepackage[]{glossaries}	        % keine Seitenangaben im Glossar, ohne separates Acronymverzeichnis
\usepackage[acronym]{glossaries}	% keine Seitenangaben im Glossar, mit separatem Acronymverzeichnis
\makeglossaries
\newacronym{ksa}{KSA}{Kantonsschule Alpenquai}
\newacronym{bla}{bla}{bla bla}
\newacronym{idn}{idn}{identification number}

\newglossaryentry{plugin}{name=Plugin, description={Ein Plugin ist ein optionales Modul, welches die Funktionalität einer Softwareanwendung erweitert}}

\newglossaryentry{qgis}{name=QGIS, description={\href{http://www.qgis.org}{QGIS} ist eine benutzerfreundliche Open-Source GIS-Software. QGIS läuft unter Linux, Unix, Mac OSX, Windows und Android. Die Entwicklung von QGIS wurde vom Kanton Solothurn stark unterstützt und gefördert. Heute umfasst QGIS eine grossen Funktionsumfang und ist mit proprietären Produkten wie ArcGIS von ESRI zumindest ebenbürtig. Eine grosse Stärke von QGIS liegt in der Unterstützung einer Vielzahl von Vektor-, Raster- und Datenbankformaten. Durch die dynamische Entwicklergemeinschaft, welche QGIS ständig ausbaut, ist QGIS jedoch für gelegentliche Anwender komplex geworden und benötigt eine Unterstützung und Begleitung.  Die Anwendung von QGIS im Verwaltungsalltag muss deshalb gezielt erfolgen und unterstützt werden}}
                     % greift auf Glossar.tex zu, worin alle Glossareinträge definiert sind

%___________________________ Optionen  _______________________________________
%\hypersetup{allcolors=black} % für den Papierausdruck
\setlength\parskip{2.5ex}     % Abstand zwischen zwei Paragraphen
\setlength\parindent{0em}     % Einzug eines neuen Paragraphen


%___________________________ Definitionen von Textbausteinen _________________
\def\BILDER{Bilder/}    % Ordner, wo alle Bilder abgelegt sind.
\def\maAutorIn{•}
\def\maKlasse{, 6•}
\def\maOrt{•}
\def\maTitel{Ein langer Titel\\ kann in kleine Stücke\\ zerlegt werden} 
\setlength{\vPositionTitel}{2.0mm} % positive Länge schiebt Titel nach oben
\setlength{\logoBindung}{0.0mm}    % positive Länge schiebt grosses Logo nach rechts
\setlength{\bildBindung}{0.0mm}    % positive Länge schiebt linken Bildrand nach rechts
\def\maTitelfarbe{\color{white}}   % black; Liste weiterer Farben: http://www.math.harvard.edu/computing/latex/color.html
\def\maTitelbild{\BILDER venice}
\def\maDate{\today}                % am Schluss das fixe Datum einsetzen!
\def\maBetreuung{•, \maFachschaft •}
\def\maFRAGE{•}
% Zitat (optional)
\renewcommand*{\dictumwidth}{.4\textwidth}
\renewcommand*{\dictumauthorformat}[1]{\normalsize (#1)}
\def\maDictum{\vfill\large\dictum[•]{\large •}}
%_____________________________________________________________________________
%_____________________________________________________________________________

\pagenumbering{Roman}
\setcounter{page}{0}
\begin{document}

%________________________ Titelseite _________________________________________
\SimpleFrontPagefalse % wird auskommentiert für selbstgestaltetes Titelbild
\ifSimpleFrontPage
	\titlehead{\Large
	\begin{center}
		\includegraphics[width=0.7\textwidth]\maTitelbild
	\end{center}
	%\centerline{\hrulefill} 
	\bigskip
	\subject{\Maturaarbeit \vspace{1cm}}
	\title{\maTitel\vspace{1.5cm}}
	\author{\maAutorIn\maKlasse}
	\date{\today}
	\publishers{\bigskip {\large \maBetr \maBetreuung}
	%\maDictum 			% Zitat kann auskommentiert werden.
	}
	\maketitle
\else   
	\AddToShipoutPicture*{\BackgroundPic}% Definition siehe KSA.sty (ab Linie 207)
\fi
\thispagestyle{empty}
%________________________ Abstract    ________________________________________
\begin{abstract}
	% !TEX root = MA.tex
\textbf{Abstract}
Gegenstand der vorliegenden Maturaarbeit ist die Beantwortung der folgenden Leitfrage: \enquote{\maFRAGE} ...
 	% Name des Files
\end{abstract}
\newpage
%________________________ Tips        ________________________________________
% !TEX encoding = UTF-8 Unicode
% !TEX root = MA.tex

\section*{Tips}
\subsection*{Cheat Sheet}
Viele gute Tips sind im \href{run:../Dokumentationen/cheat-sheet.pdf}{\ttfamily cheat-sheet.pdf} zu finden.

\subsection*{Anführungs- und Schlusszeichen, Texthervorhebungen}
Mehrfache Leerschläge            im Quelltext haben keine Auswirkungen.

Abschnitte werden durch eine leere Linie erzeugt.

Anführungs- und Schlusszeichen werden im Volksmund auch \enquote{Gänsefüsschen} genannt. 

Ein  \textbf{fett gesetzter Teil} behindert den Lesefluss. Deswegen ist ein \emph{kursiv gesetzter Text} zu bevorzugen.

Eigennamen (zum Beispiel für den Satz von \textsc{Pythagoras}), werden häufig in \textsc{Kapitälchen} gesetzt.
\subsection*{Einfache Textbausteine}

\def\FCL{Fussballclub Luzern}
Der Befehl \begin{verbatim*}\def\FCL{Fussballclub Luzern}\end{verbatim*} 
definiert die Abkürzung \begin{verbatim*}\FCL\end{verbatim*} 
als \enquote{\FCL}.

\subsection*{Tabellen}
Tabellen haben eine grosse Anfangshürde. Die Webseite \url{http://www.tablesgenerator.com} ist sehr nützlich. Tabellen können in Excel erstellt werden und dann auf der Website in \LaTeX -Quellcode umgewandelt werden.

Hier ist eine Tabelle mit Dezimaltabulator:
\begin{table}[htp]
\caption{Messwerte}
\label{tab:messwerte1}
\begin{center}
\begin{tabular}{|d{0}|d{1}|}
\hline
\text{Zeit in (s)} & \text{Temperatur in \SI{}{\celsius}} \\
\hline\hline
0 & 10.5 \\
30 & 15.7 \\
\hline
\end{tabular}
\end{center}
\end{table}

\subsection*{Bilder}
Das Logo unsrer Schule ist in Abbildung~\vref{fig:ksa} zu finden.
\begin{figure}[htb]
	\centering
		\includegraphics{\BILDER ksalpenquai} \\
		\vspace{1ex}
		\includegraphics[width=0.3\textwidth]{\BILDER ksalpenquai}
 	\caption[Logo unserer Schule]{Logo unserer Schule, in Originalgrösse und auf 0.3-fache Textweite gesetzt.\footnotemark}
  \label{fig:ksa}
\end{figure}
	\footnotetext{Beispiel für eine Fussnote in einer Bildbeschreibung.}

\textbf{Wichtiger Hinweis}: der Dateiname eines Bildes darf keine Umlaute enthalten.

\subsection*{Aus Mathematik und Physik}

Wenn ein Mensch mit \SI{1}{\meter\per\second} unterwegs ist, so beträgt seine Geschwindigkeit \SI{3600}{\meter\per\hour}. 
Das ist gleichviel wie \SI{3.6}{\kilo\meter\per\hour} oder \SI{1e9}{\nano\meter\per\second}.

Eine Schokolade kostet \FR{3.20}

Die Formel $F =  m \cdot  g$ ist den Leuten im Maturajahr mit einer Wahrscheinlichkeit von \SI{73.1}{\percent} bekannt. 

Zahlen können so dargestellt werden: \num{0.1234567}. Multipliziert mit \num{1e6} ergibt das \num{123456.7}.

Eine Gleichung ohne Nummer sieht so aus:
\begin{equation*}
c^2 =  a^2 + b^2
\end{equation*}

Eine Gleichung mit Kasten und Nummer sieht so aus:
\begin{equation}
\boxed{
	c^2 =  a^2 + b^2
}
\end{equation}

Eine andere Umgebung für mehrere Gleichungen untereinander ist hier dargestellt:
\begin{align*}
e &= \SI{1.60E-19}{\coulomb} &\text{Elementarladung} \\
m_{e} &=  \SI{9.11E-31}{\kilo\gram} &\text{Ruhemasse des Elektrons}\\
u &=  \SI{1.66e-27}{\kilo\gram} &\text{Atommassen-Einheit} \\
G&= \SI{6.67259E-11}{\meter^3\per(\kilo\gram\second^2)} &\text{Gravitationskonstante} \\
k& = \SI{8.988E9}{\newton\meter^2\per\coulomb^2}= \frac{1}{4 \pi \cdot \epsilon_0}  &\text{Elektrostatik} \\
\epsilon_0&= \SI{8.854E-12}{\coulomb^2\per(\newton\meter^2)} &\text{elektrische Feldkonstante}
\end{align*}

Dasselbe, aber mit nur teilweise nummerierten Gleichungen:
\begin{align}
e &= \SI{1.60E-19}{\coulomb} &\text{Elementarladung} \\
m_{e} &=  \SI{9.11E-31}{\kilo\gram} &\text{Ruhemasse des Elektrons}\notag\\
u &=  \SI{1.66e-27}{\kilo\gram} &\text{Atommassen-Einheit} \\
G&= \SI{6.67259E-11}{\meter^3\per(\kilo\gram\second^2)} &\text{Gravitationskonstante} \\
k& = \SI{8.988E9}{\newton\meter^2\per\coulomb^2}= \frac{1}{4 \pi \cdot \epsilon_0}  &\text{Elektrostatik} \\
\epsilon_0&= \SI{8.854E-12}{\coulomb^2\per(\newton\meter^2)} &\text{elektrische Feldkonstante}
\end{align}

Auch Vektoren können einfach dargestellt werden: 
\begin{equation*}
\vec{a} =  \frac{ \Delta \vec{v} }{ \Delta t} 
\end{equation*}
\subsection*{Geogebra}

Bilder können exportiert werden als \LaTeX -Code.

\begin{pspicture*}(-1.12,-0.72)(8.42,3.48)
\psset{xunit=1.0cm,yunit=1.0cm,algebraic=true,dimen=middle,dotstyle=o,dotsize=3pt 0,linewidth=0.8pt,arrowsize=3pt 2,arrowinset=0.25}
\multips(0,0)(0,1.0){5}{\psline[linestyle=dashed,linecap=1,dash=1.5pt 1.5pt,linewidth=0.4pt,linecolor=lightgray]{c-c}(-1.12,0)(8.42,0)}
\multips(-1,0)(1.0,0){10}{\psline[linestyle=dashed,linecap=1,dash=1.5pt 1.5pt,linewidth=0.4pt,linecolor=lightgray]{c-c}(0,-0.72)(0,3.48)}
\psaxes[labelFontSize=\scriptstyle,xAxis=true,yAxis=true,Dx=1.,Dy=1.,ticksize=-2pt 0,subticks=2]{->}(0,0)(-1.12,-0.72)(8.42,3.48)
\rput[tl](4.66,2.9){$ \SI{1}{\meter\per\second} \hat= \SI{1}{\centi\meter}$}
\end{pspicture*}

\subsection*{Chemie}
Das ist eine Reaktionsgleichung:
\begin{equation}
\ce{CO2 + C -> 2 CO} 
\end{equation}
Im Text können chemische Gleichungen wie \ce{A <--> B} ebenfalls verwendet werden\footnote{Hilfe zu Chemie findet sich im Helpfile mhchem.pdf des Pakets \enquote{mhchem}}.

Etwas komplizierter:
\begin{align*}
  \ce{RNO2 &<=>[+e] RNO2^{-.} \\
       RNO2^{-.} &<=>[+e] RNO2^2-}
\end{align*}

Falls Strukturformeln benötigt werden:\\
\url{https://www.latex-kurs.de/kurse/2017/Kurs2/Teil10/Chemie.pdf}
\begin{equation}
\chemfig{H-C(-[2]H)(-[6]H)-H} \quad\quad \chemfig{*6(=-=-=-)} \quad\quad \chemfig{H-[:52.24]\lewis{1:3:,O}-[::-104.48]H}
\end{equation}

\subsection*{Quellen}

Hier wird eine Webseite~\parencite{maKSA} zitiert. \cite{Theis2014} hat sein Buch geschrieben. Daraus ist speziell ein Fallbeispiel \cite[siehe][Seite 32]{Theis2014} interessant. Näheres ist in \url{https://en.wikibooks.org/wiki/LaTeX/Bibliographies_with_biblatex_and_biber} zu finden. \footnote{Eine Fussnote mit einer Quellenangabe \parencite[siehe][Seite 32]{Theis2014}.}

Hier wird eine URL mit Umlaut zitiert:\cite{Gärtner}.

\subsection*{Glossar, Akronyme}
Hilfe: \url{https://en.wikibooks.org/wiki/LaTeX/Glossary}.

An der \Acrlong{ksa}  ist die Abkürzung  \acrshort{ksa} geläufig.


Der erste Gebrauch von \gls{idn} sieht anders aus als der zweite Gebrauch von \gls{idn}. 

Der Gebrauch von \acrshort{idn} sieht anders aus als der  Gebrauch von \ACRshort{idn}. 

Der erste Gebrauch von \gls{bla} sieht anders aus als der zweite Gebrauch von \gls{bla}. 

Der erste Gebrauch von \gls{qgis} sieht nicht anders aus als der zweite Gebrauch von \gls{qgis}. 

          % wird auskommentiert, sobald nicht mehr gebraucht
%________________________ Inhaltsverzeichnis    ______________________________
\tableofcontents
\newpage
\pagenumbering{arabic}
%________________________ Kapitel    _________________________________________
 
% !TEX root = MA.tex
\section{Vorwort}
\label{sec:vorwort}
Nach einigen Überlegungen konnte ich die Fragestellung folgendermassen eingrenzen:
\begin{itemize}
\item \maFRAGE
\end{itemize}

\blindtext

\blindtext
		% Name des Files

% !TEX root = MA.tex
\section{Methoden}
\label{sec:methode}

		% Name des Files

% !TEX root = MA.tex
\section{Ergebnisse}


\label{sec:ergebnisse}	        % Name des Files

% !TEX root = MA.tex
\section{Diskussion und Schlussfolgerung}
\label{sec:diskussion}	        % Name des Files

% !TEX root = MA.tex
\section{Reflexion und Ausblick}
Wenigstens kann ich jetzt \LaTeX, sonst hat die Arbeit aber nichts gebracht.
\label{sec:refl-und-ausbl}
		% Name des Files

%________________________ Quellen    _________________________________________
%\renewcommand\refname{Literaturverzeichnis} 
 \printbibliography 
%________________________ Abbildungsverzeichnis    ___________________________
\BemerkungLoFig{Abbildungen ohne Quellenangaben wurden von der Autorin selbst erstellt.}
\listoffigures

%________________________ Tabellenverzeichnis    _____________________________
\BemerkungLoTab{Sämtliche Tabellen wurden vom Autor selbst erstellt.} 
\listoftables

%________________________ Redlichkeitserklärung    ___________________________
\maRedlichkeit
%________________________ Anhang    __________________________________________
\appendix\newpage
% !TEX root = MA.tex
\section*{\Huge \maAppendix}
\label{sec:anhang}
\printglossaries

\section{\maAppendix\ 1}
\label{sec:anhang1}

\section{\maAppendix\ 2}
\label{sec:anhang2}
		% Name des Files
\end{document}

